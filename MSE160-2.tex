\documentclass[a4paper,12pt]{report}
\usepackage{hyperref}

\begin{document}

\title{MSE160 Notes: Materials NTK}
\author{Eli Scott and Aman Bhargava}
\date{\today}
\maketitle

\tableofcontents

\section{Basic Properties}
\begin{itemize}
\item \textbf{Resilience}: Ability to absorb energy and recover it upon unloading (elastic energy)
\item \textbf{Tensile Strength}: Absolute peak stress on stress-strain curve.
\item \textbf{Fracture Toughness}: $K_c$, Fracture toughness (amount of energy absorbed before rupture)
\item \textbf{Polymorphism}: Ability to have more than one molecular structure.
\item \textbf{Amorphous}: Non-crystalline solid w/o long-range order.
\item \textbf{Yield Stress}: Stress that causes 0.2\% elastic deformation.
\item \textbf{Cathodic Protection}: When you use an anodic material to protect a cathodic material from corrosion.

\end{itemize}

\section{Material Design}
\begin{enumerate}
\item Choose what to min/max
\item Decide on functionality (supporting, min deflection)
\item Decide objectives
\item Decide constraints
\item Figure out free var
\item Derive function for objective and constraints
\item Eliminate free variable using constraints
\item Solve for geometric/functional/material parameters
\item Always maximize
\item Cost parameter = material index/cost per kg
\end{enumerate}


\section{Crystal Structures}
\paragraph{Theoretical Density: } $$\rho = \frac{n*A/N_a}{V_c}$$ Where $n$ is the number of atoms per unit cell, $A$ is the atomic mass, $N_a$ is avogadro's number, and $V_c$ is the volume of the unit cell.
\subsection{Simple Cubic}
\begin{itemize}
\item Packing efficiency: 52.4\%
\item Coordination number: 6
\item One atom at each corner.
\end{itemize}

\subsection{Body Centered Cubic}
\begin{itemize}
\item Packing efficiency: 68\%
\item Coordination number: 8
\item One atom in the middle and one atom at each corner.
\item Slip system: plane = $(0, 1, 1)$, direction = $<1, 1, 1>$
\end{itemize}

\subsection{Face Centered Cubic}
\begin{itemize}
\item Packing efficiency: 74\%
\item Coordination number: 12
\item Eigth of atom at corners. Half atoms on each face.
\item Slip system: plane = $(1, 1, 1)$, direction = $<1, 1, 1>$
\end{itemize}

\paragraph{Trend for coordination number in ionic crystal} $~\frac{r_{cation}}{r_{anion}}$

\subsection{Ductile-Brittle Transition}
BCC and polymers experience transition from brittle to ductile from low to high temperatures. 




\section{Strength of Crystalline Materials (Metals)}
\subsection{4 Strengthening Mechanisms: }
\begin{enumerate}
\item Alloying (larger and smaller atoms)
\item Precipitation/"particle" strengthening (relies on the fact that distortions can't easily pass through graini boundaries)
\item Cold work/work hardening
\item Grain size reduction
\end{enumerate}

\paragraph{Equation to fit work hardening stress-strain response}
$$\sigma_t = K(\epsilon_t)^n$$
Where $t$ indicates 'true' as opposed to engineering values, $K$ is a constant and $n$ is the hardening exponent.

\subsection{Conditions for Substitutional Solid Solutions}
\begin{itemize}
\item $\Delta r < 15\%$
\item Similar electronegativities
\item Same crystal structures
\item Similar valences
\end{itemize}

\paragraph{Empiric relationship between $\sigma_y$ and concentration of alloy $C$: } $sigma_y ~ \sqrt{C}$

\subsection{Resolved Shear Stress: }
$$\tau_r = \sigma_o*cos(\lambda)*cos(\phi)$$
$\lambda$ is the angle between the $\sigma_o$ and the plane, $\phi$ is the angle between the normal of the plane and ${\sigma}_o$

$\tau_{crss} = \frac{\sigma_{y}}{2}$ is the critical resolved shear stress.

\subsection{Defects}
Types of Defects:
\begin{enumerate}
\item Point defects (vacancy, self-interstitial, substitutions)
\item Linear defects (dislocations)
\item Area defects (grain bounds)
\end{enumerate}

\paragraph{Conditions for dislocation motion: } $\tau_r > \tau_{crss}$

\subsection{Equilibrium Concentration of Point Defects}
$$\frac{N_v}{N} = exp(\frac{-Q_v}{kT}$$ Where $\frac{N_v}{N}$ is the ratio of vacancies to potential vacancies (i.e. number of atoms), $Q_v$ is activiatioin energy, $k$ is Boltzmann's constant, and $T$ is temperature.

\subsection{Fatigue}
Fatigue is cyclic stressing. Here are the main parameters:
\begin{enumerate}
\item $S$: amplitude of cyclic stress
\item $\sigma_0$: mean stress
\item Frequency of stressing
\end{enumerate}

$$\frac{da}{dN} = \Delta K^{m}$$
Where a = half crack length, N = number of cycles, $\Delta K ~ \Delta \sigma * \sqrt(a)$, m is a constant

\textbf{$S_{fat}$ is the minimum $S$ for fatigue behavior.}


\subsection{Hall-Petch Equation}
Gives you $\sigma_y$ from grain size.
$$\sigma_y = sigma_0 + k_y * d^{-1/2}$$
Where $\sigma_0$ and $k_y$ are constants and $d$ is the diameter of the grain size.





\section{C R A C K}
\subsection{Griffith's Formula: $\sigma_m$}
Crack tip stress. $$\sigma_m = 2*\sigma_o*\sqrt{a/p_t}$$
Where $p_t$ is the radius of curvature, a is half crack length, and $\sigma_o$ is the applied stress.

\paragraph{Propagation criterion: } When $\sigma_m > \sigma_c$

\subsection{$\sigma_c$}
Critical stress, $$\sigma_c = \sqrt{2E\gamma_s}{\pi*a}$$ Where $E$ is modulus of elasticity, $\gamma_s$ is the specific surface energy (add $\gamma_p$ = plastic deformation energy if ductile), $a$ is half crack length.

Basically just the stress that cracks can withstand before propagating. 

\subsection{$K$}
At crack failure, $$K_c = Y*\sigma*sqrt{\pi*a}$$ Where $K_c$ is fracture toughness of material, $Y$ is a constant, and $a$ is the half crack width.

Two failure cases here: either cracks are too large or $\sigma$ is too large.

\section{Electrical Properties}

\subsection{4 Ways to Increase Resistivity?}
\begin{enumerate}
\item Increase temperaturre
\item Increase grain boundaries, dislocations
\item Add impurities
\item Add vacancies
\end{enumerate}

\paragraph{Calculating resistance of a sample: } $R = \frac{L}{A*\sigma}$

\subsection{P vs. N-Type Doping}
P-type brings the acceptor band down significantly. N-type brings the donor band up. P-type results in more free holes, N-type results in more free electrons. 


\paragraph{Current Density} = $J = I/A$ = current/cross sectional area.
\paragraph{Electric Field Potential} = $E = V/L$

\subsection{New Ohm's Law}
$$J = \sigma*E$$

\subsection{Calculating $\sigma$}
$$\sigma = n|e|\mu_e + p|e|\mu_h$$
Where $n$ is the number of mobile electrons, $p$ is the number of holes, $\mu_e$ is electron mobility and $\mu_h$ is the hole mobility.

\paragraph{Voltage threshold for semiconductors: $2eV$}


\end{document}
